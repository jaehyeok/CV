% LaTeX file for resume 
% This file uses the resume document class (res.cls)

\documentclass[margin]{res} 
% the margin option causes section titles to appear to the left of body text 
\textwidth=5.2in % increase textwidth to get smaller right margin
%\usepackage{helvetica} % uses helvetica postscript font (download helvetica.sty)
%\usepackage{newcent}   % uses new century schoolbook postscript font 
\usepackage{hyperref}
\usepackage{setspace}
\hypersetup{colorlinks=true}

\begin{document} 

\name{\LARGE \CV \\ \\
      \Large \hspace{0.8cm} Jae Hyeok Yoo\\[12pt]} % the \\[12pt] adds a blank line after name
 
\address{ Department of Physics, UC San Diego \\ 
          Mayer Hall, Room 5531 \\ 
          9500 Gilman Drive \\ 
          La Jolla, CA 92093 USA  \\
          (858) 900-7473 \\
          \href{mailto:jay013@physics.ucsd.edu}{jay013@physics.ucsd.edu} } 

\setstretch{1.1}

\begin{resume} 

% FIXME  
%\begin{itemize}
%\item look for typos   
%\item Does it makes sense to include the SUSY papaers I contributed to 
%      by electron charge flip measurement, efficiency measurements, ... 
%\end{itemize}
% FIXME  

%%%%% Education %%%%%%%%%%%%%%%%%%%%%%%%%%%%%%%%%%%%%%%%%%%%%%%%%%%%%%%
\section{Education} 
\textbf{University of California, San Diego, La Jolla, CA} \hfill  Dec 2013(expected) \\
Ph.D. in Physics \\
 \begin{itemize} \itemsep -2pt  % reduce space between items
 \item Thesis Title : \textit{Evidence for a New Boson Decaying into Two W Bosons
 and Study on its Spin-parity in the Fully Leptonic Final States with the  
 4.9 and 19.5 $fb^{-1}$ of Data at the Center-of-mass Energy of 7 and 8 TeV with CMS Detector}
 \item Thesis advisor : Professor Frank W\"urthwein   
 \end{itemize}

\textbf{Korea University, Seoul, South Korea} \hfill Feb 2009 \\
M.S. in Physics 
 \begin{itemize} \itemsep -2pt  % reduce space between items
 \item Thesis Title : \textit{A Study of the Process $e^+e^- \rightarrow c\bar{c}c\bar{c}$} 
 \item Thesis advisor : Professor Eunil Won  
 \end{itemize}
\textbf{Korea University, Seoul, South Korea} \hfill Feb 2007 \\
B.S. in Physics\\

%%%%% Research Experience %%%%%%%%%%%%%%%%%%%%%%%%%%%%%%%%%%%%%%%%%%%%%%%%%%%%%%%
\section{Research Experience}
\textbf{Graduate Research Assistant}                             \hfill 2009 - present     \\
\textit{University of California, San Diego, Department of Physics}                        \\
Advisor :  Professor Frank W\"urthwein                                                       
\begin{itemize} \itemsep -2pt  % reduce space between items
 \item For $H\rightarrow W^+W^-\rightarrow2l2\nu$, developed a new analysis method  
       using 2D fit  
       that enhanced the search sensitivity by 25 \% at $m_{\mathrm{H}}$ = 125 GeV, 
       which has been selected as the main analysis method since HCP 2012 conference 
 \item Studied the spin-parity nature of the new boson using a hypothesis separation test 
       between SM Higgs and a Graviton-like spin-2 model at $m_{\mathrm{H}}$ = 125 GeV
 \item Validated the fit model of the major backgrounds using dedicated data events
 \item Performed extensive studies on the post-fit behavior of nuisance parameters 
       and the background shape/normalizations 
 \item For di-lepton SUSY analyses, measured the lepton selection and trigger efficiencies, 
       and the electron charge mis-measurement rate 
\end{itemize}

\textbf{Graduate Research Assistant}           \hfill 2007 - 2009        \\
\textit{Korea University, Department of Physics}       \\
Advisor : Professor Eunil Won                                                                 
 \begin{itemize} \itemsep -2pt  % reduce space between items
 \item Measured the cross section upper limit of four-charm production at Belle   
\\
\end{itemize}

%\textbf{Undergraduate Research Assistant}                 \hfill 2006 - 2007        \\
%\textit{Korea University, Department of Physics}       \\
%Advisor : Professor Eunil Won                                                                
% \begin{itemize} \itemsep -2pt  % reduce space between items
% \item Measured the mean muon lifetime 
% \end{itemize}

%%%%% Academic Honors %%%%%%%%%%%%%%%%%%%%%%%%%%%%%%%%%%%%%%%%%%%%%%%%%%%%%%%
\section{Honors and Awards} 
Travel award for DPF2013                                \hfill Aug 2013 \\
Brain Korea 21 scholarship                              \hfill 2007 - 2009 \\
Scholarships for academic excellence, Korea University  \hfill 2005 - 2006 \\


%%%%% Presentations %%%%%%%%%%%%%%%%%%%%%%%%%%%%%%%%%%%%%%%%%%%%%%%%%%%%%%%
\section{Presentations } 
\textit{``Evidence for a Particle Decaying to $W^+W^-$ in the Fully Leptonic Final State 
in a Standard Model Higgs Boson Search''}, \\ 
Parallel session, DPF 2013, Santa Cruz, CA, USA  
\hfill{Aug 2013}
\\
\\
\textit{``Evidence for a Particle Decaying to $W^+W^-$ in the Fully Leptonic Final State 
in a Standard Model Higgs Boson Search''}, \\
Poster session, Lepton-Photon 2013, San Francisco, CA, USA  
\hfill{Jun 2013}
\\
\\
\textit{``Mean Muon Lifetime Measurement''}, \\
Parallel session, Korea Physical Society Conference, Daegu, South Korea  
\hfill{Oct 2006}

%%%%% Publication %%%%%%%%%%%%%%%%%%%%%%%%%%%%%%%%%%%%%%%%%%%%%%%%%%%%%%%
%\section{Publications}

%Please see the List of Publication in the separate document.

%%%%% Teaching and Other Experience %%%%%%%%%%%%%%%%%%%%%%%%%%%%%%%%%%%%%%%%%%%%%%%%%%%%%%%
%\section{Teaching Experience}
%\textbf{Graduate Teaching Assistant}                          \hfill 2009 - 2010        \\
%\textit{University of California, San Diego, Department of Physics}                       
% \begin{itemize} \itemsep -2pt  % reduce space between items
% \item Freshmen and sophomore physics labs for non-physics majors 
% \end{itemize}
%\textbf{Graduate Teaching Assistant}                      \hfill 2007 - 2009        \\
%\textit{Korea University, Department of Physics}      
% \begin{itemize} \itemsep -2pt  % reduce space between items
% \item Electronics lab for junior physics majors 
% \item Freshmen physics for science majors
% \item Introductory physics course for non-physics majors (taught in English) 
% \end{itemize}
%Korean Augmentation To the US Army (mandatory military service)    \hfill 2002 - 2004        \\ 

%%%%% References %%%%%%%%%%%%%%%%%%%%%%%%%%%%%%%%%%%%%%%%%%%%%%%%%%%%%%%
%\newpage
\section{References}
\textbf{Dr. Frank W\"urthwein}, %                                           \\  
Professor, Department of Physics, UC San Diego                  \\
Mayer Hall 3310 \\
9500 Gilman Drive \\
La Jolla, CA 92093-0319 \\
(858) 822-3219 \\
\href{fkw@ucsd.edu}{fkw@ucsd.edu} \\
\\
\textbf{Dr. Claudio Campagnari}, %                                          \\ 
Professor, Department of Physics, UC Santa Barbara              \\
5119 Broida Hall \\
%Physics Department \\
%University of California \\
Santa Barbara, CA 93106 \\
(805) 893-7567 \\
\href{claudio@physics.ucsb.edu}{claudio@physics.ucsb.edu} \\
\\
\textbf{Dr.  Christoph Paus}, %                                                         \\
Professor, Department of Physics, MIT     \\
Massachusetts Institute of Technology  \\
77 Massachusetts Avenue, 24-509 \\
Cambridge, MA 02139 \\
(617) 258-0314  \\ 
\href{paus@mit.edu}{paus@mit.edu} \\
\\
\textbf{Dr. Avraham Yagil}, %                                                   \\ 
Professor, Department of Physics, UC San Diego                  \\
Mayer Hall 5512  \\ 
9500 Gilman Drive \\
La Jolla, CA 92093-0319 \\
(858) 534-9504 \\
\href{ayagil@physics.ucsd.edu}{ayagil@physics.ucsd.edu} \\
\\
\textbf{Dr. Guillelmo Gomez Ceballos Retuerto}, %                                               \\
Scientist, Department of Physics, MIT    \\
32-4-A11 \\
CERN CH-1211 Geneva 23 \\
Switzerland \\
+41 22 76 74450 \\
\href{guillelmo.gomez.ceballos@cern.ch}{guillelmo.gomez.ceballos@cern.ch}

\newpage
\textbf{ \textcolor{red}{ 
The next few pages are not part of CV, but if you are interested in the details of my work,
please take a look. The list is in reverse chronological order.}}

%%%%% Research Experience %%%%%%%%%%%%%%%%%%%%%%%%%%%%%%%%%%%%%%%%%%%%%%%%%%%%%%%
\newpage
\section{Full List of Research Experience}
\textbf{------------------------------------ 
                Research at CMS 
        ------------------------------------} 
\\
\\
\textbf{$\mathbf{H\rightarrow WW \rightarrow 2l2\nu}$} 
        \hfill Jan 2012 - present 
    \\   
    \\
      $\bullet\quad$  \textbf{One of the main contributors to the analysis for the three conference results
      (ICHEP 2012, HCP 2012, and Moriond 2013) and for the final publication which is in progress.} 
      Carried out the whole analysis from making analysis ntuples to delivering  the 
      final numbers (background estimations, limits, significances, and spin-parity).
    \\
    \\
      $\bullet\quad$ \textbf{Developed, implemented, and validated a new analysis method(2D analysis) 
      using 2-dimensional binned templates of $m_{\mathrm{T}}$ and $m_{ll}$, which was selected 
      as the main analysis method in $e\mu$ final state for the $H\rightarrow WW \rightarrow 2l2\nu$ at CMS.}
      At $m_{\mathrm{H}}=125$ GeV,
      the new method enhanced the search sensitivity by 25 \% compared to the existing 
      analysis method using BDT. The new method allowed easier interpretation of the result 
      by virtue of using physical variables opposed to the BDT method. 
      In addition, because same background templates with same selections can be used 
      for multiple $m_{\mathrm{H}}$ hypotheses, implementation is simpler 
      than the BDT method which needed separate selections at different $m_{\mathrm{H}}$. 
      I worked on defining range/binning/selections for the new 
      analysis to achieve better search sensitivity and to ensure reliability of the method. 
    \\
    \\
    $\bullet\quad$  \textbf{Performed an analysis on the spin-parity nature of the new boson} 
      by hypothesis separation test between SM Higgs and Graviton-like spin-2 model using 
      2D templates of $m_{\mathrm{T}}$ and $m_{ll}$. 
      Constructed the \textit{p.d.f.}s  using pseudo-data 
      and measured the expected/observed separations. 
    \\  
    \\
    $\bullet\quad$  \textbf{Performed various fit validation studies on data.} The 2D analysis relies on 
      simulation for the background shapes, so it is critical to ensure that our fit 
      model fits the data correctly. I performed various tests to validate the fit
      model using dedicated data control regions populated by different background processes 
      such as WW, Top, W+jets and W$\gamma$(*). These studies required deep understanding
      on the fit model and the statistical machinery, not only conceptually but also
      technically.
    \\ 
    \\
    $\bullet\quad$  \textbf{Performed extensive studies on the post-fit behavior of nuisance parameters 
      and background shape/normalization.}
      Studied impact of various nuisance parameters to the search sensitivity,  
      correlations between them, changes of nuisance parameters by fit, 
      changes of background shape/normalization by fit.
      Wrote various scripts to study pulls and correlations of nuisances, 
      some of which have been used by other people/groups.  
    \\
    \\
    $\bullet\quad$  \textbf{Made significant contribution to the development of MVA-based Drell-Yan suppression technique.}
      The $ee/\mu\mu$ channel is less sensitive than $e\mu$ channel due to
      tight MET cut to suppress the contribution from Drell-Yan process. I performed 
      several versions of BDT training(different input variables, 
      BDT parameters tuning, and different samples) to attain better performance. 
      I also studied over-training and correlation with the main analysis BDT that was 
      designed to distinguish signal from the main(WW/Top) backgrounds.  
      (BDT was used for $ee/\mu\mu$ category when the new method was developed.)
    \\
    \\
    $\bullet\quad$  \textbf{Validated MVA-based jet identification technique.} This technique was developed 
      to mitigate degradation of jet identification performance due to Pile-Up. 
      I implemented this technique in the analysis framework and validated its performance.
    \\
    \\
    $\bullet\quad$  \textbf{Was responsible for producing analysis ntuples. }
      This required a deep understanding on the basic objects and selections 
      as well as an ability to implement them in the analysis framework.
      Working toward discovery in summer 2012, 
      it was very important to deliver results at a very short time scale. One of the most 
      challenging tasks was to process as much data as possible and to make ntuples 
      for analysis as quickly as possible. I was one of the two people responsible for 
      making analysis ntuples for our group until ICHEP 2012. 
    \\    
    \\
    $\bullet\quad$  \textbf{Re-analyzed 7 TeV data using 2D method in the $e\mu$ final states.} 
    The search sensitivity improved by 20\% at $m_{\mathrm{H}}$ = 125 GeV.  
    \\  
    \\
    $\bullet\quad$  Gave a number of talks at regular CMS HWW group meetings 
      and gave the pre-approval talk for HCP 2012 where  
      the new analysis method was presented in public for the first time.
    \\
    \\
    $\bullet\quad$  Am the author of the CMS internal note(AN) for 2D method and 
    made significant contributions to the 4 ANs and 4 public notes(PAS) which are 
    listed in the ``Public Notes" section.   
    \\
    \\
\textbf{Feasibility study on $t\bar{t}W/Z$ cross section measurement in the tri-lepton final state} 
        \hfill Sep 2011 - Dec 2011 
     \\
     \\
     Performed a feasibility study for observation of SM rare process $t\bar{t}W/Z$ 
     in 5 $fb^{-1}$ at $\sqrt{s}=8$ TeV.  
     In the course of the study, I found a couple of bugs in the LHE file of the signal samples 
     which was used for signal MC production. The bugs were reported to the production team 
     and fixed in the next production of the samples.
     \\
     \\
\textbf{Monte Carlo sample generation} 
       \hfill Sep 2011 - Dec 2011 
     \\                
     \\                
     Generated signal MC for the same-sign SUSY analysis. Produced samples via grid using 
     generated LHE files as an input. 
     Wrote scripts to map an LHE file to a grid job, which was the main challenge.  
     \\                
     \\                
\textbf{Tri-lepton exercise with WZ cross section measurement} 
        \hfill Jun 2011 - Aug 2011 
     \\
     \\
     Performed an analysis exercise in preparation for a tri-lepton analysis such as $t\bar{t}W/Z$ 
     cross section measurement. Basic selections were exploited and cross section was measured.
     The measured cross section was consistent with the CMS public result. 
     \\
     \\
\textbf{Measurement of L1 Trigger Efficiency}  
        \hfill Dec 2010 - Jan 2011 
     \\
     \\
     Measured efficiency of the electron L1 trigger requirements
     with respect to the offline selections using Tag-And-Probe method. Result was presented at 
     CMS Trigger Studies Group(TSG) meeting.  
     \\
     \\
\textbf{Measurement of lepton selection/trigger efficiency} 
        \hfill Oct 2010 - May 2011 
     \\
     \\
     Measured offline lepton selection efficiencies and trigger efficiencies for 
     the same-sign and opposite-sign di-lepton SUSY analyses using Tag-And-Probe method. 
     In 2010 the single-lepton trigger menu changed rapidly as the instantaneous 
     luminosity increased dramatically. So, measuring individual trigger efficiency
     was very challenging. My group developed a trigger model that uses the efficiency 
     of a soup of single-lepton triggers and calculates the per-event trigger efficiency. 
     I measured the efficiency of the soup of triggers and validated the trigger model. 
     \\
     \\
\textbf{Fake rate study for SS di-lepton analysis}
      \hfill{July 2010 - Sep 2010} 
     \\
     \\
     Performed MC closure tests. Studied dependence of fake rates(FR) on 
     $p_T$ and $\eta$ of leptons, away jet $p_T$ threshold, and b-tagging requirement for the away jet
     on both MC and data. For data, studied dependence of FR on triggers as well.  
     \\
     \\
\textbf{Development of condor job monitoring}
      \hfill{Jun 2010 - Sep 2010} 
     \\
     \\
     Built a tool using C++ and bash to monitor the status of condor jobs at the gatekeepers 
     to prevent jobs held too long and distribute them to other gatekeepers that have 
     free nodes. 
     \\
     \\
\textbf{Measurement of electron charge-flip rate} 
      \hfill Jan 2010 - Jun 2010 
      \\ 
      \\ 
      Produced single electron gun samples and measured electron charge mis-measurement rate 
      for the same-sign analysis. Performed the measurements with different CMSSW releases 
      and validated the results from one against the other.
      \\
      \\
\textbf{Processing data ntuples for my research group} 
      \hfill Mar 2011  - present 
      \\ 
      \\
      Have been in charge of processing some of the data samples and contributed to development/maintenance 
      of the tool. 
      \\
      \\
      \\
\textbf{--------------------------------- 
                Research before CMS 
        ---------------------------------}
     \\
     \\
\textbf{Measurement of cross section for the process $e^+e^- \rightarrow c\bar{c}\bar{c}c$} 
      \hfill Jan 2007 - Feb 2009 
      \\
      \\
      Studied the process $e^+e^- \rightarrow c\bar{c}\bar{c}c$ at $\sqrt{s} = 10.52$ GeV at Belle. 
      Used CompHEP to generate the signal sample and fed the matrix element result to the 
      Belle analysis software for the hadronization and the detector simulation. 
      Studied $e^+e^- \rightarrow D^0D^0+X$ where $D^0$ decays to $K^-\pi^+$.
      Exploited a number of cut variables to suppress $q\bar{q}$ backgrounds 
      and applied the likelihood ratio obtained from the super Fox-Wolfram moment 
      that reduced the background significantly. 
      Used sideband subtraction method to extract signal component. 
      Analyzed 68 $fb^{-1}$ of data and set the cross section upper limit.
      \\
      \\
%      Link to the thesis : \href{http://hep.korea.ac.kr/~jhyoo/thesis.pdf}
%                                {http://hep.korea.ac.kr/~jhyoo/thesis.pdf} 
%      \\ 
%      \\ 
\textbf{Simulation of energy deposit in a silicon strip detector}  
      \hfill Oct 2006 - Dec 2006
      \\
      \\
       Performed a Geant4 simulation of energy deposit in the silicon strip sensor of 
       width 0.38 mm with 30 MeV proton beam.  
      \\
      \\
\textbf{Measurement of the mean muon lifetime}  
      \hfill Mar 2006 - Dec 2006
      \\
      \\
      Measured the mean muon lifetime as a undergraduate senior research program.
      %The experimental setup was composed of a scintillator, a photo-multiplier tube, 
      %a discriminator, a FPGA using VHDL, a FADC, and a computer. 
      Fixed errors in the main part of the VHDL code from which trigger signals were generated. 
      Fitted the lifetime distribution using ROOT to extract the mean muon lifetime. 
      %Measured mean lifetime was $2.17 \pm 0.03~\mu s$ 
      %which was consistent with the PDG value, $2.19703 \pm 0.00004~\mu s$ within uncertainty. 
      %This result was presented at the fall conference of Korea Physical Society in 2006 as 
      %an oral talk.




\end{resume} 
\end{document} 



